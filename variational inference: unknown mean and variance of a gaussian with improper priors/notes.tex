\documentclass[11pt]{article}

\usepackage[utf8]{inputenc}
\usepackage{amsmath}
\usepackage{amsthm}
\usepackage{amsfonts}
\usepackage{bm}
\usepackage[T1]{fontenc}

\title{Variational Inference: Unknown mean and variance of a Gaussian}
\date{}
\begin{document}
\maketitle

Máme data $\mathcal{D} = \{x_1, \dots, x_N\}$, která jsou nezávislá a pochází z
normálního rozdělení s neznámou střední hodnotou a přesností.

Log likelihood parametrů $\mu, \tau$ je
\begin{align}
p(\mathcal{D} \mid \mu, \tau) &=
    \prod_{i=1}^N \sqrt{\frac{\tau}{2\pi}}
              \exp \left(\frac{-\tau(x_i - \mu)^2}{2}\right)
\\
\log p(\mathcal{D} \mid \mu, \tau) &=
    \sum_{i=1}^N
        \frac{1}{2} \log \frac{\tau}{2\pi} - \frac{\tau(x_i - \mu)^2}{2}
\\
&=
    \frac{N}{2} \log \tau
    - \sum_{i=1}^N \frac{\tau (x_i - \mu)^2}{2}
    + C
\\
&=
    \frac{N}{2} \log \tau
    - \frac{\tau}{2}
    \left(
        N(\mu - \bar x)^2
        + \sum_{i=1}^N (x_i - \bar x)^2
    \right)
    + C
\end{align}

Pravděpodobnostní model reprezentuje sdruženou pravděpodobnost $p(\bm{X},
\bm{Z})$ a my potřebujeme nalézt aproximaci aposteriorní distribuce $p(\bm{Z}
\mid \bm{X})$ a také pravděpodobnost pozorování $p(\bm{X})$.

\begin{align}
\log p(\bm{X}) &=
    \mathcal{L}(q) + \mathrm{KL}(q \| p)
\\
\mathcal{L}(q) &=
    \int q(\bm{Z}) \log
        \left(
            \frac{p(\bm{X}, \bm{Z})}
                 {q(\bm{Z})}
        \right) \mathrm{d}\bm{Z}
\\
\mathrm{KL}(q \| p) &=
    - \int q(\bm{Z}) \log
        \left(
            \frac{p(\bm{Z} \mid \bm{X})}
                 {q(\bm{Z})}
        \right) \mathrm{d}\bm{Z}
\end{align}

Aposteriorní distribuci faktorizujeme
\begin{equation}
q(\bm{Z}) = \prod_i^M q_i(\bm{Z}_i)
\end{equation}

Likelihood s faktorizovanou distribucí pak bude
\begin{align}
\mathcal{L}(q) &=
    \int \left( \prod_i q_i\right)
    \left(
        \log p(\bm{X}, \bm{Z})
        - \sum_i \log q_i
    \right) \mathrm{d}\bm{Z}
\\
&= \int q_j
    \left(
        \int \log p(\bm{X}, \bm{Z})
        \prod_{i \ne j} q_i ~
            \mathrm{d}\bm{Z}_i
    \right) \mathrm{d}\bm{Z}_j -
\\
&- \int q_j \log q_j ~ \mathrm{d}\bm{Z}_j + C
\\
&= \int q_j
    \log \tilde{p}(\bm{X}, \bm{Z}_j)
    \mathrm{d}\bm{Z}_j -
    \int q_j \log q_j \mathrm{d}\bm{Z}_j
    + C \label{eq:lkl}
\\
\log \tilde{p}(\bm{X}, \bm{Z}_j) &= 
    \mathbb{E}_{i \ne j} [ \log p(\bm{X}, \bm{Z})] + C
\end{align}

Předpokládejme, že všechny $\{q_{i \ne j}\}$ jsou zafixované a maximizujeme
$\mathcal{L}(q)$ pro distribuci $q_j(\bm{Z}_j)$.
To provedeme jednoduše když vezmeme v úvahu, že rovnice (\ref{eq:lkl}) je
negativní Kullback-Leibler divergence mezi $q_j(\bm{Z}_j)$ a $\tilde{p}(\bm{X},
\bm{Z}_j)$. Takže maximalizace (\ref{eq:lkl}) je to samé jako minimalizace KL
divergence a minimum nastane když $q_j(\bm{Z}_j) = \tilde{p}(\bm{X},
\bm{Z}_j)$. Obecná formule pro optimální řešení $q_j^*(\bm{Z}_j)$ tedy bude

\begin{equation}
\log q_j^*(\bm{Z}_j) =
    \mathbb{E}_{i \ne j} [ \log p(\bm{X}, \bm{Z})] + C
\label{eq:truelog}
\end{equation}

Nyní k příkladu, aposterioriní pravděpodobnost budeme aproximovat faktory:
\begin{equation}
    q(\mu, \tau) = q_\mu (\mu) q_\tau (\tau)
\end{equation}

Optimální faktory získáme z rovnice (\ref{eq:truelog}):

\begin{align}
\log q_\mu^*(\mu) &=
    \mathbb{E}_\tau [
        \log p(\mathcal{D} \mid \mu, \tau)
        + \log (p(\mu))
        + \log (p(\tau))
    ]
\\
&= \mathbb{E}_\tau \bigg[
    \frac{N}{2} \log \tau
    - \frac{\tau}{2}
    \Big(
        N(\mu - \bar x)^2
        + \sum_i^N (x_i - \bar x)^2
    \Big) +
\\
& + \log \frac{1}{\sigma_\mu}
    + \log \frac{1}{\tau}
    + C
    \bigg]
\nonumber
\\
&= -\frac{N\mathbb{E}_\tau [\tau]}
         {2}
    (\mu - \bar{x})^2 + C
\end{align}

Nyní už je vidět, že $q_\mu^*(\mu)$ je ve formě normálního rozdělení
\begin{align}
q_\mu^*(\mu) &= N(\mu \mid \bar x, \lambda^{-1})
\\
\lambda &= N\mathbb{E}_\tau[\tau]
\end{align}

Pro $q_\tau^*(\tau)$ budeme postupovat obdobně:
\begin{align}
\log q_\tau^*(\tau) &= 
	\mathbb{E}_\mu [
		\log p(\mathcal{D} \mid \mu, \tau) 
		+ \log p(\mu)
		+ \log p(\tau)
	]
\\
&= \mathbb{E}_\mu \bigg[
	\frac{N}{2} \log(\tau)
	- \frac{\tau}{2} \Big(
    \sum_{i=1}^N (x_i - \mu)^2
	\Big) +
\\
& 	+ \log \frac{1}{\sigma_\mu}
	+ \log \frac{1}{\tau}
	+ C
	\bigg]
\nonumber
\\
&= \frac{N-2}{2} \log(\tau)
	- \frac{\tau}{2} \Big(
		\mathbb{E}_\mu \Big[
            \sum_{i=1}^N (x_i^2 - 2 x_i \mu + \mu^2)
        \Big]
	\Big)
	+ C
\\
&= \frac{N-2}{2} \log(\tau)
	- \frac{\tau}{2} \Big(
    \sum_{i=1}^N x_i^2 - 2 \mathbb{E}_\mu[\mu] \sum_{i=1}^N x_i + N
    \mathbb{E}_\mu[\mu^2])
	\Big)
	+ C
\end{align}

Pokud výsledný tvar srovnáme s předpisem pro Gamma rozložení
\begin{equation}
Gamma(\tau \mid a, b) = b^2 \frac{1}{\Gamma(a)} \tau^{a-1} \exp(-b\tau)
\end{equation}
s parametry
\begin{align}
a &= \frac{N}{2}
\\
b &= \frac{1}{2} \Big(
    \sum_{i=1}^N x_i^2 - 2 \mathbb{E}_\mu[\mu] \sum_{i=1}^N x_i + N
    \mathbb{E}_\mu[\mu^2])
	\Big)
\end{align}

Nyní můžeme oba kroky střídat a iterovat dokud nedojdeme k pevnému bodu. Při
iteraci využijeme toho, že pro nalezení nové hodnoty $q_\mu(\mu)$ nám stačí
znát střední hodnotu $\tau$, což pro Gamma rozdělení není problém. Pro nalezení
nové hodnoty $q_\tau(\tau)$ nám zase stačí znát $\mathbb{E}_\mu[\mu]$ a
$\mathbb{E}_\mu[\mu^2]$
















\end{document}
